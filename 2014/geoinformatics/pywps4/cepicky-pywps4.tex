\documentclass[xcolor=dvipsnames,notes=hide]{beamer}
%\documentclass[notes=only]{beamer}
%\documentclass[xcolor=dvipsnames, notes=hide]{beamer} 
\usepackage{ulem}
\usepackage{tabularx}
\usepackage{xcolor,colortbl}
\usepackage{pgfpages}

\newcommand{\mc}[2]{\multicolumn{#1}{c}{#2}}
\definecolor{Gray}{gray}{0.85}
\definecolor{LightGreen}{rgb}{1,0.88,1}

\newcolumntype{a}{>{\columncolor{Gray}} X }
\newcolumntype{b}{>{\columncolor{white}} X }

\usepackage{default}
\usepackage{pgfpages} %This is needed for notes presentation!
\setbeameroption{show notes on second screen=left}
\usepackage{soul}
\usepackage[czech]{babel}
\usepackage[utf8]{inputenc}
\usepackage{hyperref}
\hypersetup{colorlinks,linkcolor=NavyBlue,urlcolor=NavyBlue}
\usepackage{times}



\pgfdeclareimage[height=1.cm]{conference-logo}{images/dasenka.jpg}
\logo{\pgfuseimage{conference-logo}}

\title{PyWPS4}

\subtitle {Status report}

\author[J. Čepický] % (optional, use only with lots of authors)
{Jáchym~Čepický\inst{1}}
% - Give the names in the same order as the appear in the paper.
% - Use the \inst{?} command only if the authors have different
%   affiliation.

\institute % (optional, but mostly needed)
{
  \inst{1}%
  \url{http://les-ejk.cz}, \url{http://geosense.cz}\\
}
  
% - Use the \inst command only if there are several affiliations.
% - Keep it simple, no one is interested in your street address.

\date[Geoinformatics 2014] % (optional, should be abbreviation of conference name)
{Geoinformatics 2014}
% - Either use conference name or its abbreviation.
% - Not really informative to the audience, more for people (including
%   yourself) who are reading the slides online



\begin{document}

%\begin{abstract}
%\end{abstract}

\begin{frame}
\titlepage

\note{Dobrý den\\
jmenuji se Jáchym Čepický}
\end{frame}

\begin{frame}\begin{center}
\includegraphics[width=0.4\textwidth]<1>{images/pywps_logo.png}

\note<1>{dnes tady mluvím za vývojový tým projektu PyWPS}
\note<2>{
    PyWPS je projekt vyvíjený od roku 2006, jedná se o implementaci standardu OGC
    WPS na straně serveru v jazyce Python. Stávající verze je vyvíjena pod licencí
    GNU/GPL, nová verze pod licencí BSD
}
\end{center}\end{frame}

\begin{frame}\begin{center}
\includegraphics[width=1\textwidth]<1>{images/python.jpg}

\note<1>{Už jsem zmínil to, že PyWPS je vyvíjen v jazyce Python. Mezi ostatními
implementacemi vyniká především rychlostí instalace a konfigurace. Instalace
proběhne během několika minut. Psaní skriptů je také velice jednoduché.}

\includegraphics[width=1\textwidth]<2>{images/grass.jpg}
\note<2>{PyWPS obsahuje od začátku podporu pro skripty v GRASS GIS}

\includegraphics[width=0.4\textwidth]<3>{images/gdal.png}
\note<3>{GDAL/OGR nebo}

\includegraphics[width=0.4\textwidth]<4>{images/r.png}
\note<4>{R. Proč jsme se rozhodli pro přepsání}

\end{center}\end{frame}


\begin{frame}\begin{center}
    \includegraphics<2>[height=7cm]{images/desert.jpg}
\note<1>{ Už jsem to tady říkal vloni. V roce 2006 vypadala krajina geopythonu jinak }

\note<2>{
    \begin{itemize}
        \item GRASS neměl žádné rozhraní pro jazyk python
        \item Python 2.2 byla aktuální verze
        \item Prakticky žádná podpora pro parserování XML souborů
        \item Chyběly další knihovny pro práci s OGC
        \item Nejúžívanější formát byl ESRI Shapfile, mluvilo se o GML, GeoTIFF
    \end{itemize}
}

\includegraphics<3>[height=7cm]{images/prales.jpg}
\note<3>{
    \begin{itemize}
        \item Python 3
        \item GRASS má nativní vazbu do Pythonu, stejně jako další knihovny
        \item Nové projekty lxml, owslib, werkzeug
    \item Nové formáty - GeoJSON, TopoJSON (kde je dneska KML, \dots)
    \end{itemize}
}
\end{center}\end{frame}


\begin{frame}\begin{center}
    \url{https://github.com/jachym/pywps-4/issues/milestones}
    \note{
Na uvedené adrese najdete roadmap, která je
stále ve vývoji. Máme to rozděleno na PyWPS 4.0.0 a PyWPS 4.1.0 verze.
}
\end{center}\end{frame}

\begin{frame}\begin{center}
    PyWPS 4.0: Co máme
    \begin{itemize}
        \item Validators
        \item Server implementation based on werkzeug
        \item IOHandler
        \item File Storage
    \end{itemize}
    \note{
        \begin{itemize}
            \item Validators - gml, shp, 4 stupně
            \item Server implementation based on werkzeug - populární knihonva na
                tvorbu serverových aplikací
            \item IOHandler univerzální objekt pro převod dat mezi objekty stream,
                data, in memory object a další
            \item File Storage - obsluha výsledných souborů
        \end{itemize}
    }
\end{center}\end{frame}

\begin{frame}\begin{center}
Do PyWPS 4.1.0 by měly patřit nové vlastnosti:
    
\begin{itemize}
    \item Výstupy přes GeoServer, MapServer, QGIS Server
    \item Administrativní RESTové rozhranní
\end{itemize}
    \note{
        PyWPS 4.1 by mělo obsahovat
\begin{itemize}
    \item Výstupy přes GeoServer, MapServer, QGIS Server
    \item Administrativní RESTové rozhranní
    \item Database storage, external service storage (FTP, Dropbox, ...)
\end{itemize}
    }
\end{center}\end{frame}

\begin{frame}\begin{center}
Co nás brzdí

\begin{itemize}
    \item Čas
    \item Peníze
\end{itemize}

\note{
    Nedostatek času a peněz. Momentálně nikdo nemá v popisu práce pracovat na PyWPS.
    Chybí nám sponzoři, dělníci. Něco se na obzoru vrbí, ale zatím nic
    konkrétního. Pro open source projekt jako je tento je nedostatek pracovníků
    kritický. Na aktuálním kódu pracují ve svém velice volném čase asi 3 lidi.
~\\

    Získali jsme jednoho studenta do Google Summer of Code Project - ve
    spolupráci s OSGeo.
}
\end{center}\end{frame}

\begin{frame}\begin{center}
    \begin{itemize}
\item Intevation GmbH
\item Help Service - Remote Sensing
\item Deutsche Bundesstiftung Umwelt
    \end{itemize}

    \note{S tím souvisí i naši minulí sponzoři, byl bych rád,kdybych mohl příští
    rok zde ukazovat delší seznam našich podporovatelů, a hlavně aktuální.
Každpáně bez těchto firem a institucí by PyWPS nebylo dnes tam kde je.
}
\end{center}\end{frame}

\section*{Závěr}
\begin{frame}\begin{center}
    Dotazy?\\

    \vfill

    \url{http://pywps.intevation.org}

    \vfill
    \includegraphics[width=0.15\textwidth]{images/cc.png}

\note{Tato prezentace je samozřejmě uvolněna pod licencí Creative Commons,
Uveďte autora. \\
~\\
Děkuji Vám za pozornost, kterou jste mi věnovali a jsem připraven
čelit vašim dotazům, máte-li nějaké}
\end{center}\end{frame}


\end{document}
