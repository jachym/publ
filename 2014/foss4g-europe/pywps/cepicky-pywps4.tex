\documentclass[xcolor=dvipsnames,notes=hide]{beamer}
%\documentclass[notes=only]{beamer}
%\documentclass[xcolor=dvipsnames, notes=hide]{beamer} 
\usepackage{ulem}
\usepackage{tabularx}
\usepackage{xcolor,colortbl}
\usepackage{pgfpages}

\newcommand{\mc}[2]{\multicolumn{#1}{c}{#2}}
\definecolor{Gray}{gray}{0.85}
\definecolor{LightGreen}{rgb}{1,0.88,1}

\newcolumntype{a}{>{\columncolor{Gray}} X }
\newcolumntype{b}{>{\columncolor{white}} X }

\usepackage{default}
\usepackage{pgfpages} %This is needed for notes presentation!
\setbeameroption{show notes on second screen=left}
\usepackage{soul}
\usepackage[czech]{babel}
\usepackage[utf8]{inputenc}
\usepackage{hyperref}
\hypersetup{colorlinks,linkcolor=NavyBlue,urlcolor=NavyBlue}
\usepackage{times}



\pgfdeclareimage[height=1.cm]{conference-logo}{images/event-logo.png}
\logo{\pgfuseimage{conference-logo}}

\title{PyWPS}

\subtitle {Status report}

\author[J. Čepický] % (optional, use only with lots of authors)
{Jáchym~Čepický\inst{1}}
% - Give the names in the same order as the appear in the paper.
% - Use the \inst{?} command only if the authors have different
%   affiliation.

\institute % (optional, but mostly needed)
{
  \inst{1}%
  \url{http://les-ejk.cz}, \url{http://geosense.cz}\\
}
  
% - Use the \inst command only if there are several affiliations.
% - Keep it simple, no one is interested in your street address.

\date[FOSS4G-E 2014] % (optional, should be abbreviation of conference name)
{FOSS4G-Europe 2014}
% - Either use conference name or its abbreviation.
% - Not really informative to the audience, more for people (including
%   yourself) who are reading the slides online



\begin{document}

%\begin{abstract}
%\end{abstract}

\begin{frame}
\titlepage

\note{Hallo\\
my name is Jáchym Čepický}
\end{frame}

\begin{frame}\begin{center}
\includegraphics[width=0.4\textwidth]{images/pywps_logo.png}

\note<1>{today, I'm talking here on behalf of the PyWPS development team}
\note<2>{
    PyWPS development started in 2006, just early enough, to be first time
    presented at first FOSS4G in Lausanne. It is implementation of OGC WPS
    standard on the server side and it's written in Python programming language.
    Curent version is distributed under the GNU/GPL license. New PyWPS (we call
    it 4) is developed under MIT license.
}
\end{center}\end{frame}

\begin{frame}\begin{center}
\includegraphics[width=1\textwidth]<1>{images/python.jpg}

\note<1>{I already metioned, PyWPS is written in Python.
   Among other implementations of the standard, PyWPS is known for it's
   simplicity, regarding installation and setup. Instalation is matter of
   several minutes. After that, user must write her own scripts, which are then
   interpreted as processes.
}

\includegraphics[width=1\textwidth]<2>{images/grass.jpg}
\note<2>{Since beginning, PyWPS contained support for GRASS GIS modules.}

\includegraphics[width=0.4\textwidth]<3>{images/gdal.png}
\note<3>{GDAL/OGR or}

\includegraphics[width=0.4\textwidth]<4>{images/r.png}
\note<4>{R}

\end{center}\end{frame}
%%%%%%%%%%%%%%%%%%%%%%%%%%%%%%%%%%%%%%%%%%%%%%%%%%%%%%%%%%%%%%%%%%%%%%%%%%%%
\begin{frame}\begin{center}
    PyWPS-4
\note{
    So, why did we start to rewrite PyWPS from scratch?
}
\end{center}\end{frame}
%%%%%%%%%%%%%%%%%%%%%%%%%%%%%%%%%%%%%%%%%%%%%%%%%%%%%%%%%%%%%%%%%%%%%%%%%%%%


\begin{frame}\begin{center}
    \includegraphics<2>[height=7cm]{images/desert.jpg}
\note<1>{In 2006, the world of GeoPython was different}

\note<2>{
    \begin{itemize}
        \item There was no GRASS Python API
        \item Python was in 2.2 version
        \item For working with xmls, you had to do everything manually
        \item There were no libraries, which would help you to deal with OGC
            Services
        \item Most used format was ESRI Shapefile and we started to talk about
            GML.
    \end{itemize}
}

\includegraphics<3>[height=7cm]{images/prales.jpg}
\note<3>{
    Today, we have
    \begin{itemize}
        \item Python 3
        \item Native GRASS Python API
        \item new projects, like lxml, owslib, werkzeug
        \item new formats - GeoJSON, TopoJSON (anybody heard about KML, recently? \dots)
    \end{itemize}
}
\end{center}\end{frame}


\begin{frame}\begin{center}
    \url{https://github.com/jachym/pywps-4/issues/milestones}
    \note{
On this URL, we started to creat roadmap. Currently, we are distinguishing
between PyWPS 4.0 and 4.1.
}
\end{center}\end{frame}

\begin{frame}\begin{center}
    PyWPS 4.0: What do we have
    \begin{itemize}
        \item Validators
        \item Server implementation based on werkzeug
        \item IOHandler
        \item File Storage
    \end{itemize}
    \note{
        \begin{itemize}
            \item Validators - gml, shp, 4 stupně
            \item Server implementation based on werkzeug - popular library for
                server-side apps creation
            \item IOHandler is universal objekt for transparent transformation
                of data, beteween data stream, file object, in memory object
                nad other representations.
            \item File Storage -- output files hanlder
        \end{itemize}
    }
\end{center}\end{frame}

\begin{frame}\begin{center}

   Plans for PyWPS 4.1:

\begin{itemize}
    \item Data outputs via GeoServer, MapServer and/or QGIS MapServer
    \item Administrative REST API
\end{itemize}
    \note{
        PyWPS 4.1 should contain
\begin{itemize}
    \item Outpu via GeoServer, MapServer, QGIS Server
    \item Administrative REST API
    \item Database storage, external service storage (FTP, Dropbox, ...)
\end{itemize}
    }
\end{center}\end{frame}

%%%%%%%%%%%%%%%%%%%%%%%%%%%%%%%%%%%%%%%%%%%%%%%%%%%%%%%%%%%%%%%%%%%%%%%%%
\begin{frame}\begin{center}

\includegraphics<1>[width=0.8\textwidth]{images/break.jpg}
\includegraphics<2>[width=0.8\textwidth]{images/time_monay.jpg}

\note<1>{what does break us}

\note<2>{
    The team is curently of of time and there are no external resources
    currently, to move faster forward.\\
    ~\\
    We have to confess, that for open source project about this size, lack of
    resources is critical. We are able to maitain current version of PyWPS
    fixing, but heavy works on PyWPS 4 are impossible.
}
\end{center}\end{frame}

%%%%%%%%%%%%%%%%%%%%%%%%%%%%%%%%%%%%%%%%%%%%%%%%%%%%%%%%%%%%%%%%%%%%%%%%%
\begin{frame}\begin{center}

\includegraphics<1>[width=0.8\textwidth]{images/gsoc.png}

\note{
    This year, we had luck and we got 4 interesting proposals for Google Summer
    of code. As result, we have now one student working on process chaining on
    current version of PyWPS and we are looking forward to port her work into
    PyWPS 4.
}
\end{center}\end{frame}

%%%%%%%%%%%%%%%%%%%%%%%%%%%%%%%%%%%%%%%%%%%%%%%%%%%%%%%%%%%%%%%%%%%%%%%%%
\begin{frame}\begin{center}

\includegraphics<1>[width=0.8\textwidth]{images/tudor.jpg}

\note{
    Yearlier this year, thanks to COST framework, PyWPS had change to meet at
    code sprint at Henri Tudor research center in Luxembourg. It was join event
    with 52North WPS and we
    hope, next year the WPS comming together will be even bigger, and our
    collegues from Zoo or Geoserver or other projects will join us as well.
}
\end{center}\end{frame}
%%%%%%%%%%%%%%%%%%%%%%%%%%%%%%%%%%%%%%%%%%%%%%%%%%%%%%%%%%%%%%%%%%%%%%%%%

\begin{frame}\begin{center}
    \begin{itemize}
\item Intevation GmbH
\item Help Service - Remote Sensing
\item Deutsche Bundesstiftung Umwelt
\item Netmar project
\item COST Framework
    \end{itemize}

\note{I would like to thank to existing and past sponsors of PyWPS and encourage
    new comming sponsors, to help with the development. The companies and
    projects supported the project with either hardware or man-power, so PyWPS
    could be at the spot, where it is now.
}
\end{center}\end{frame}

\section*{Conclusion}
\begin{frame}\begin{center}
    Questions?\\

    \vfill

    \url{http://pywps.intevation.org}

    \vfill
    \includegraphics[width=0.15\textwidth]{images/cc.png}

\note{
    Thank you
}
\end{center}\end{frame}


\end{document}
